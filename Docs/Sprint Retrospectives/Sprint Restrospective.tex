\documentclass[10pt,onecolumn]{witseiepaper}
\usepackage{KJN}
\usepackage{graphicx}
\usepackage{url}

\newcommand{\ttt}{\texttt}

\title{ELEN4011 - Sprint Retrospectives}

\author{}
\thanks{School of Electrical \& Information Engineering, University of the
Witwatersrand, Private Bag 3, 2050, Johannesburg, South Africa}


\begin{document}

\maketitle
\pagestyle{plain}
\setcounter{page}{1}

\section*{SPRINT 1}
\subsection*{Attendees:}
Jared Ping
\subsection*{Agenda:} 
To critically analyse the sprint. Calculate the sprint velocity.

\subsection*{Minutes of meeting:}
Date: 31/09/2018 \\
Time: 19:00

\subsubsection*{What went right:}
\begin{enumerate}
	\item Met expected deadlines of sprint
	\item Agile methodologies were correctly followed
	\item All developer stories assigned to the sprint milestone were completed
	\item An efficient code review process was created and maintained for the duration of the sprint
	\item The use of daily stand-ups to track sprint progress
	\item CI pipeline configured using Travis CI
\end{enumerate}

\subsubsection*{What went wrong:}
\begin{enumerate}
	\item Logic based pull requests were opened without any tests
	\item No other developers to contact when developer required assistance
	\item Unsuccessful configuration of coveralls
\end{enumerate}

\subsubsection*{What needs improvement:}
\begin{enumerate}
	\item Weekly meeting to improve scrum tactics. This will ensure that the developer is well versed in agile techniques to accommodate a more streamlined process
	\item A minimum of one tests must be implemented for each logic based developer story 
	\item Implementation of tests from all levels of the test pyramid
	\item More rigorous pull request reviews. Any changes should be required, rather than suggested
	\item Improve structure of daily stand-ups to ensure that assistance is readily available 
\end{enumerate}

\subsubsection*{Velocity:}
Each developer story was assigned an effort rating between 1 and 3 based on the size of the story. Using these efforts, the sprint velocity was calculated to be 23. The ideal velocity for the next sprint is therefore 23.
\newpage
\section*{SPRINT 2}
\subsection*{Attendees:}
Jared Ping

\subsection*{Agenda:} 
To critically analyse the sprint and calculate the sprint velocity.

\subsection*{Minutes of meeting:}
Date: 09/10/2018 \\
Time: 19:00

\subsubsection*{What went right:}
\begin{enumerate}
	\item All but one developer story assigned to the sprint were completed
	\item Met expected deadlines of sprint.
	\item The use of daily stand-ups to track sprint progress
	\item Sprint board went through an iteration to cater for stakeholder feedback after first sprint
	\item Unit tests and Acceptance tests have been implemented and automated in the CI pipeline through Travis CI
\end{enumerate}

\subsubsection*{What went wrong:}
\begin{enumerate}
	\item Current sprint velocity is less than previous sprint velocity
	\item Lots of time was consumed in waiting for required student data
	\item Story relating to student data was not implemented due to time in which data was received
	\item Unsuccessful configuration of coveralls
\end{enumerate}

\subsubsection*{What needs improvement:}
\begin{enumerate}
	\item Code coverage must be tested using Coveralls
	\item Comprehensive tests required to adhere to the testing pyramid
\end{enumerate}

\subsubsection*{Velocity:}
The velocity for this sprint was 17.

\end{document}